\chapter{Структура FILE}

Описание структуры \texttt{FILE} приведено в Листинге \ref{lst:fileh}

\listingfile{FILE.h}{fileh}{C}{Файл \texttt{/usr/include/bits/types/FILE.h}}{}

Описание структуры \texttt{\_IO\_FILE} приведено в Листингах \ref{lst:iofileh} -- \ref{lst:iofileh-3}

\listingfile{struct_FILE.h}{iofileh}{C}{Файл \texttt{/usr/include/bits/types/struct\_FILE.h}, Часть 1}{linerange={1-31}}

\listingfile{struct_FILE.h}{iofileh-2}{C}{Файл \texttt{/usr/include/bits/types/struct\_FILE.h}, Часть 2}{linerange={33-88},firstnumber=33}

\clearpage

\listingfile{struct_FILE.h}{iofileh-3}{C}{Файл \texttt{/usr/include/bits/types/struct\_FILE.h}, Часть 3}{linerange={89-120},firstnumber=89}

\chapter{Первая программа}

\section{Коды программ}

\listingfile{prog1.c}{prog-01}{C}{Открытие одного и того же файла несколько раз для чтения в одном процессе}{}

\imgw{1}{h!}{\textwidth}{Программа с одним процессом}

\clearpage

\listingfile{prog1_thread.c}{prog-02-1}{C}{Открытие одного и того же файла для чтения в нескольких потоках}{}

\clearpage

\imgw{1-1}{h!}{\textwidth}{Открытие одного и того же файла для чтения в нескольких потоках}

\section{Схема взаимодействия структур}

\imgw{schema-1}{h!}{\textwidth}{Схема взаимодействия структур}

\clearpage
\section{Анализ полученного результата}
Функция \texttt{open()} создает новый файловый дескриптор файла (открытого только на чтение) "data.txt"{}, запись в системной таблице открыт файлов. Эта запись регистрирует смещение в файле и флаги состояния файла.
Функция \texttt{fdopen()} создает указатели на структуру \texttt{FILE}. Поле \texttt{\_fileno} содержит дескриптор, который вернула функция \texttt{fopen()}.
Функция \texttt{setvbuf()} явно задает размер буффера в 20 байт и меняет тип буферизации (для \texttt{fs1} и \texttt{fs2}) на полную.
При первом вызове функции \texttt{fscanf()} в цикле (для \texttt{fs1}), \texttt{buff1} будет заполнен полностью -- первыми 20 символами (буквами алфавита). \texttt{f\_pos} в структуре \texttt{struct\_file} открытого файла увеличится на 20.
При втором вызове \texttt{fscanf()} в цикле (для \texttt{fs2}) буффер \texttt{buff2} будет заполнен оставшимися 6 символами (начиная с \texttt{f\_pos}).
В цикле поочередно выводятся символы из \texttt{buff1} и \texttt{buff2}.
Все это справедливо и для многопоточной реализации: в потоке, который первый получит квант, первый вызов \texttt{fscanf()} заполнит буфер и увеличит \texttt{f\_pos} в структуре \texttt{struct\_file} открытого файла, а оставшиеся 6 символов будут записаны в буфер, находящийся в другом потоке.

\clearpage

\chapter{Вторая программа}

\section{Коды программ}

\listingfile{prog2.c}{prog-02}{C}{Открытие одного и того же файла несколько раз для чтения в одном процессе}{}

\imgw{2}{h!}{\textwidth}{Открытие одного и того же файла несколько раз для чтения в одном процессе}

\clearpage

\listingfile{prog2_thread.c}{prog-02-1}{C}{Открытие одного и того же файла несколько раз для чтения в двух потоках}{}

\imgw{2-1}{h!}{\textwidth}{Открытие одного и того же файла несколько раз для чтения в двух потоках}

\clearpage

\section{Схема взаимодействия структур}

\imgw{schema-2}{h!}{1\textwidth}{Схема взаимодействия структур}

\section{Анализ полученного результата}
Функция \texttt{open()} создает файловые дескрипторы, два раза для одного и того же файла, поэтому в программе существует две различные \texttt{struct file}, но ссылающиеся на один и тот же \texttt{struct inode}.
Из-за того что структуры \texttt{struct file} разные, посимвольная печать просто дважды выведет содержимое файла в формате <<AAbbcc...>> (в случае однопроцессной реализации).
В случае многопоточной реализации, вывод второго потока начнется позже (нужно время, для создание этого потока) и символы начнут выводится позже (см. рис. \ref{img:2-1}).

\clearpage

\chapter{Третья программа}

\section{Коды программ}

\listingfile{prog3.c}{prog-03}{C}{Открытие одного и того же файла несколько раз для записи в одном процессе}{}

\imgw{3}{h!}{\textwidth}{Открытие одного и того же файла несколько раз для записи в одном процессе}

\clearpage

\listingfile{prog3_thread.c}{prog-03-1}{C}{Открытие одного и того же файла несколько раз для записи в двух потоках}{}

\imgw{3-1}{h!}{0.9\textwidth}{Открытие одного и того же файла несколько раз для записи в двух потоках}

\clearpage
\section{Схема взаимодействия структур}

\imgw{schema-3}{h!}{1\textwidth}{Схема взаимодействия структур}

%\clearpage

\section*{Анализ полученного результата}
Файл открывается на запись два раза, с помощью функции \texttt{fopen()}.
Функция \texttt{fprintf()} предоставляет буферизованный вывод - буфер создается без нашего вмешательства.
Изначально информация пишется в буфер, а из буфера в файл если произошло одно из событий: буффер полон, вызвана функция \texttt{fclose()}, вызвана функция \texttt{fflush()}.
В случае нашей программы, информация в файл запишется в результате вызова функция \texttt{fclose()}.
Из-за того \texttt{f\_pos} независимы для каждого дескриптора файла, запись в файл будет производится с самого начала.
Таким образом, информация записанная при первом вызове \texttt{fclose()} будет потеряна в результате второго вызова \texttt{fclose()} (см. рис. \ref{img:3-1}).
В многопоточной реализации результат аналогичен - с помощью \texttt{pthread\_join} мы дожидаемся вызова \texttt{fclose()} для \texttt{f2} в отдельном потоке и далее вызываем \texttt{fclose()} для \texttt{f1}.

Для исключения проблемы потери данных нужно открывать файл в режиме добавления - \texttt{O\_APPEND}. Тогда операция записи в файл становится атомарной и перед каждым вызовом \texttt{write}, смещение в файле устанавливается в конец файла.
