\chapter*{Введение}
\addcontentsline{toc}{chapter}{Введение}
\textbf{Цель работы:} построение доверительных интервалов для математичского ожидания и дисперсии нормальной случайной величины.
\textbf{Содержание работы: }
\begin{enumerate}
	\item Для выборки объема n из нормальной генеральной совокупности X реализовать в виде программы на ЭВМ
	\begin{itemize}
		\item[а)] вычисление точечных оценок $\stackrel{\wedge}{\mu}(\overrightarrow{x_n})$ и $S^2(\overrightarrow{x_n})$ математического ожидания MX и дисперсии DX соответсвтенно;
		\item[б)] вычисление нижней и верхней границ $\underline{\mu}(\overrightarrow{x_n})$, $\overline{\mu}(\overrightarrow{x_n})$ для $\gamma$-доверительного интервала для математического ожидания MX;
		\item[в)] вычисление нижней и верхней границ $\underline{\sigma}(\overrightarrow{x_n})$, $\overline{\sigma}(\overrightarrow{x_n})$ для $\gamma$-доверительного интервала для  дисперсии DX;
	\end{itemize}
	\item вычислить $\stackrel{\wedge}{\mu}$ и $S^2$ для выборки из индивидуального варианта;
	\item для заданного пользователем уровня доверия $\gamma$ и $N$ – объема выборки из индивидуального варианта:
	\begin{itemize}
		\item[а)] на координатной плоскости $Oyn$ построить прямую
		$y = \stackrel{\wedge}{\mu}(\overrightarrow{x}_N)$, также графики функций
		$y = \stackrel{\wedge}{\mu}(\overrightarrow{x}_n)$,
		$y = \underline{\mu}(\overrightarrow{x}_n)$ и
		$y = \overline{\mu}(\overrightarrow{x}_n)$
		как функций объёма $n$ выборки, где $n$ изменяется от 1 до $N$;
		\item[б)] на другой координатной плоскости $Ozn$ построить прямую
		$z = S^2(\overrightarrow{x}_N)$, также графики функций
		$y = S^2(\overrightarrow{x}_n)$,
		$y = \underline{\sigma}^2(\overrightarrow{x}_n)$ и
		$y = \overline{\sigma}^2(\overrightarrow{x}_n)$
		как функций объёма $n$ выборки, где $n$ изменяется от 1 до $N$.
	\end{itemize}
\end{enumerate}