\chapter*{Введение}
\addcontentsline{toc}{chapter}{Введение}
\textbf{Цель работы:} построение гистограммы и эмпирической функции рапределения.
\textbf{Содержание работы: }
\begin{enumerate}
	\item Для выборки объема n из генеральной совокупности X реализовать в виде программы на ЭВМ
	\begin{itemize}
		\item[а)] вычисление максимального значения $M_{max}$ и минимального значения $M_{min}$;
		\item[б)] размаха $R$ выборки;
		\item[в)] вычисление оценок $\stackrel{\wedge}{\mu}$ и $S^2$ математического ожидания MX и дисперсии DX;
		\item[г)] группировку значений выборки в $m=[log_2 n]+2$ интервала;
		\item[д)] построение на одной координатной плоскости гистограммы и графика функции плотности распределения вероятностей нормальной случайной величины с математическим ожиданием $\stackrel{\wedge}{\mu}$ и дисперсией $S^2$;
		\item[е)] построение на другой координатной плоскости графика эмпирической функции распределения и функции распределения нормальной случайной величины математическим ожиданием $\stackrel{\wedge}{\mu}$ и дисперсией $S^2$.
	\end{itemize}
	\item Провести вычисления и построить графики для выборки из индивидуального варианта.
\end{enumerate}