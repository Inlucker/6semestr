\StructuredChapter{Введение}
Известный французский писатель и мыслитель А. Сент-Экзюпери, автор красивой сказки о Маленьком принце, оценивая значимость общения в человеческой жизни, определил его как «единственную роскошь, которая есть у человека». Реальность и необходимость общения определены совместной деятельностью людей. Именно в процессе общения и только в общении может проявиться сущность человека.

Межличностные отношения - это отношения с близкими нам людьми; это отношения между родителями и детьми, мужем и женой, братом и сестрой. Конечно, близкие личные отношения не ограничиваются кругом семьи, в таких отношениях часто находятся люди, живущие вместе под влиянием различных обстоятельств. \cite{3}

Общий фактор в этих отношениях - это различного рода чувства привязанности, любви и преданности, а также желание сохранить эти отношения. Если Ваш босс осложняет Вашу жизнь, Вы можете с ним распрощаться; если продавец в магазине не уделил Вам должного внимания, Вы туда больше не пойдете; если сотрудник(ца) поступает по отношению к Вам нелояльно, Вы предпочтете по возможности не общаться с ним (с ней) и т. д. 

Но если возникают неприятности между нами и близкими нам людьми, это обычно приобретает для нас первостепенное значение.Много ли людей приходит к психологу по причине не сложившихся отношений со своим парикмахером? С другой стороны, мы видим очень много людей, ищущих консультаций и помощи в домашних и семейных, коллективных неурядицах. 

Проблемы, касающиеся межличностных отношений уже несколько веков не только не теряют своей актуальности, но становятся все более важными для многих социально-гуманитарных наук. Анализируя межличностные отношения и возможности достижения в нем взаимопонимания, можно объяснить многие социальные проблемы развития общества, семьи и отдельной личности. Являясь неотъемлемым атрибутом жизни человека, межличностное отношение играет большую роль во всех сферах жизнедеятельности. При этом качество межличностных отношений зависит от общения, от уровня достигнутого понимания. 

Роль общения в межличностном отношении, несмотря на возросший к ней интерес ряда социально-гуманитарных наук, является все же не достаточно изученной. Поэтому выбор темы моей работы обусловлен следующими моментами: 
\begin{enumerate}
	\item[1.] Необходимостью четкого выделения категории общения из области взаимосвязанных категорий отношения; 
	\item[2.] Попыткой структурирования межличностного отношения по уровням общения. 
	\item[3.] Потребностью общества в разрешении межличностных и внутриличностных конфликтов, связанных с непониманием. 
\end{enumerate}

Целью данной работы является осмысление роли общения межличностном отношении, а так же в попытке структурирования межличностных отношений по уровням общения. 
