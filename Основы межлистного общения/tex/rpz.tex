%% Преамбула TeX-файла

% 1. Стиль и язык
\documentclass[utf8x, 14pt]{G7-32} % Стиль (по умолчанию будет 14pt)

% Остальные стандартные настройки убраны в preamble.inc.tex.
\usepackage{csvsimple}

\bibliography{biblio}
\lstset{frame=none, tabsize=4}

% Настройки листингов.
\ifPDFTeX
% 8 Листинги

\usepackage{listings}

% Значения по умолчанию
\lstset{
  basicstyle= \footnotesize,
  breakatwhitespace=true,% разрыв строк только на whitespacce
  breaklines=true,       % переносить длинные строки
%   captionpos=b,          % подписи снизу -- вроде не надо
  inputencoding=koi8-r,
  numbers=left,          % нумерация слева
  numberstyle=\footnotesize,
  showspaces=false,      % показывать пробелы подчеркиваниями -- идиотизм 70-х годов
  showstringspaces=false,
  showtabs=false,        % и табы тоже
  stepnumber=1,
  tabsize=4,              % кому нужны табы по 8 символов?
  frame=single
}

% Стиль для псевдокода: строчки обычно короткие, поэтому размер шрифта побольше
\lstdefinestyle{pseudocode}{
  basicstyle=\small,
  keywordstyle=\color{black}\bfseries\underbar,
  language=Pseudocode,
  numberstyle=\footnotesize,
  commentstyle=\footnotesize\it
}

% Стиль для обычного кода: маленький шрифт
\lstdefinestyle{realcode}{
  basicstyle=\scriptsize,
  numberstyle=\footnotesize
}

% Стиль для коротких кусков обычного кода: средний шрифт
\lstdefinestyle{simplecode}{
  basicstyle=\footnotesize,
  numberstyle=\footnotesize
}

% Стиль для BNF
\lstdefinestyle{grammar}{
  basicstyle=\footnotesize,
  numberstyle=\footnotesize,
  stringstyle=\bfseries\ttfamily,
  language=BNF
}

% Определим свой язык для написания псевдокодов на основе Python
\lstdefinelanguage[]{Pseudocode}[]{Python}{
  morekeywords={each,empty,wait,do},% ключевые слова добавлять сюда
  morecomment=[s]{\{}{\}},% комменты {а-ля Pascal} смотрятся нагляднее
  literate=% а сюда добавлять операторы, которые хотите отображать как мат. символы
    {->}{\ensuremath{$\rightarrow$}~}2%
    {<-}{\ensuremath{$\leftarrow$}~}2%
    {:=}{\ensuremath{$\leftarrow$}~}2%
    {<--}{\ensuremath{$\Longleftarrow$}~}2%
}[keywords,comments]

% Свой язык для задания грамматик в BNF
\lstdefinelanguage[]{BNF}[]{}{
  morekeywords={},
  morecomment=[s]{@}{@},
  morestring=[b]",%
  literate=%
    {->}{\ensuremath{$\rightarrow$}~}2%
    {*}{\ensuremath{$^*$}~}2%
    {+}{\ensuremath{$^+$}~}2%
    {|}{\ensuremath{$|$}~}2%
}[keywords,comments,strings]

% Подписи к листингам на русском языке.
\renewcommand\lstlistingname{Листинг}
\renewcommand\lstlistlistingname{Листинги}

\else
\usepackage{local-minted}
\fi

% Полезные макросы листингов.
% Любимые команды
\newcommand{\Code}[1]{\textbf{#1}}


\begin{document}

\frontmatter % выключает нумерацию ВСЕГО; здесь начинаются ненумерованные главы: реферат, введение, глоссарий, сокращения и прочее.

% Стиль титульного листа и заголовки
%\include{00-title}

\setcounter{page}{3}

\tableofcontents

%\StructuredChapter{Перечень сокращений и обозначений}

В настоящем отчете о НИР применяют следующие сокращения и обозначения

\noindent ГП --- графический процессор

\noindent СУБД --- система управления базами данных

\noindent ЦПУ --- центральное процессорное устройство

\noindent OLAP --- online analytical processing

\noindent OLTP --- online transaction processing

\StructuredChapter{Введение}
Известный французский писатель и мыслитель А. Сент-Экзюпери, автор красивой сказки о Маленьком принце, оценивая значимость общения в человеческой жизни, определил его как «единственную роскошь, которая есть у человека». Реальность и необходимость общения определены совместной деятельностью людей. Именно в процессе общения и только в общении может проявиться сущность человека.

Межличностные отношения - это отношения с близкими нам людьми; это отношения между родителями и детьми, мужем и женой, братом и сестрой. Конечно, близкие личные отношения не ограничиваются кругом семьи, в таких отношениях часто находятся люди, живущие вместе под влиянием различных обстоятельств \cite{3}. 

Общий фактор в этих отношениях - это различного рода чувства привязанности, любви и преданности, а также желание сохранить эти отношения. Если Ваш босс осложняет Вашу жизнь, Вы можете с ним распрощаться; если продавец в магазине не уделил Вам должного внимания, Вы туда больше не пойдете; если сотрудник(ца) поступает по отношению к Вам нелояльно, Вы предпочтете по возможности не общаться с ним (с ней) и т. д. 

Но если возникают неприятности между нами и близкими нам людьми, это обычно приобретает для нас первостепенное значение.Много ли людей приходит к психологу по причине не сложившихся отношений со своим парикмахером? С другой стороны, мы видим очень много людей, ищущих консультаций и помощи в домашних и семейных, коллективных неурядицах. 

Проблемы, касающиеся межличностных отношений уже несколько веков не только не теряют своей актуальности, но становятся все более важными для многих социально-гуманитарных наук. Анализируя межличностные отношения и возможности достижения в нем взаимопонимания, можно объяснить многие социальные проблемы развития общества, семьи и отдельной личности. Являясь неотъемлемым атрибутом жизни человека, межличностное отношение играет большую роль во всех сферах жизнедеятельности. При этом качество межличностных отношений зависит от общения, от уровня достигнутого понимания. 

Роль общения в межличностном отношении, несмотря на возросший к ней интерес ряда социально-гуманитарных наук, является все же не достаточно изученной. Поэтому выбор темы моей работы обусловлен следующими моментами: 
\begin{enumerate}
	\item[1.] Необходимостью четкого выделения категории общения из области взаимосвязанных категорий отношения; 
	\item[2.] Попыткой структурирования межличностного отношения по уровням общения. 
	\item[3.] Потребностью общества в разрешении межличностных и внутриличностных конфликтов, связанных с непониманием. 
\end{enumerate}

Целью данной работы является осмысление роли общения межличностном отношении, а так же в попытке структурирования межличностных отношений по уровням общения. 


\mainmatter % это включает нумерацию глав и секций в документе ниже

\chapter{Межличностные отношения}
\section{Место и природа межличностных отношений}
В социально-психологической литературе высказываются различные точки зрения на вопрос о том, где «расположены» межличностные отношения, прежде всего относительно системы общественных отношений. Природа межличностных отношений может быть правильно понята, если их не ставить в один ряд с общественными отношениями, а увидеть в них особый ряд отношений, возникающий внутрикаждого вида общественных отношений, не вне их. 

Природа межличностных отношений существенно отличается от природы общественных отношений: их важнейшая специфическая черта — эмоциональная основа. Поэтому межличностные отношения можно рассматривать как фактор психологического «климата» группы. Эмоциональная основа межличностных отношений означает, что они возникают и складываются на основе определенных чувств, рождающихся у людей по отношению друг к другу. В отечественной школе психологии различаются три вида, или уровня эмоциональных проявлений личности: аффекты, эмоции и чувства. Эмоциональная основа межличностных отношений включает все виды этих эмоциональных проявлений \cite{4}. 

Отношения между людьми не складываются лишь на основе непосредственных эмоциональных контактов. Сама деятельность задает и другой ряд отношений, опосредованных ею. Поэтому-то и является чрезвычайно важной и трудной задачей социальной психологии одновременный анализ двух рядов отношений в группе: как межличностных, так и опосредованных совместной деятельностью, т.е. в конечном счете стоящих за ними общественных отношений.

Все это ставит очень остро вопрос о методических средствах такого анализа. Традиционная социальная психология обращала преимущественно свое внимание на межличностные отношения, поэтому относительно их изучения значительно раньше и полнее был разработан арсенал методических средств. Главным из таких средств является широко известный в социальной психологии метод социометрии, предложенный американским исследователем Дж. Морено, для которого она есть приложение к его особой теоретической позиции. Хотя несостоятельность этой концепции давно подвергнута критике, методика, разработанная в рамках этой теоретической схемы, оказалась весьма популярной.

Таким образом, мы можем сказать, что межличностные отношения рассматриваются как фактор психологического «климата» группы. Но для диагностики межличностных и межгрупповых отношений в целях их изменения, улучшения и совершенствования применяется социометрическая техника, основоположником которой является американский психиатр и социальный психолог Дж. Морено. 

\section{Сущность межличностных отношений}
Межличностные отношения — это совокупность связей, складывающихся между людьми в форме чувств, суждений и обращений друг к другу \cite{28}. 

Межличностные отношения включают:
\begin{enumerate}
	\item восприятие и понимание людьми друг друга;
	\item межличностную привлекательность (притяжение и симпатия);
	\item взаимодействие и поведение (в частности, ролевое).
\end{enumerate}

Компоненты межличностных отношений:

\begin{enumerate}
	\item когнитивный компонент — включает в себя все познавательные психические процессы: ощущения, восприятие, представление, память, мышление, воображение. Благодаря этому компоненту происходит познание индивидуально-психологических особенностей партнеров по совместной деятельности и взаимопонимание между людьми. Характеристиками взаимопонимания являются:
		\subitem a) адекватность — точность психического отражения воспринимаемой личности;
		\subitem б) идентификация — отождествление индивидом своей личности с личностью другого индивида;
	\item эмоциональный компонент — включает положительные или отрицательные переживания, возникающие у человека при межличностном общении с другими людьми:
		\subitem a) симпатии или антипатии;
		\subitem б) удовлетворенность собой, партнером, работой и т.д.;
		\subitem в) эмпатия — эмоциональный отклик на переживания другого человека, который может проявляться в виде сопереживания (переживания тех чувств, которые испытывает другой), сочувствия (личностного отношения к переживаниям другого) и соучастия (сопереживание, сопровождаемое содействием);
	\item поведенческий компонент — включает мимику, жестикуляцию, пантомимику, речь и действия, выражающие отношения данного человека к другим людям, к группе в целом. Он играет ведущую роль в регулировании взаимоотношений. Эффективность межличностных отношений оценивается по состоянию удовлетворенности— неудовлетворенности группы и ее членов.
\end{enumerate}

Виды межличностных отношений:
\begin{enumerate}
	\item производственные отношения — складываются между сотрудниками организаций при решении производственных, учебных, хозяйственных, бытовых и др. проблем и предполагают закрепленные правила поведения сотрудников по отношению друг к другу. Разделяются на отношения:
		\subitem a) по вертикали — между руководителями и подчиненными;
		\subitem б) по горизонтали — отношения между сотрудниками, имеющими одинаковый статус;
		\subitem б) по диагонали — отношения между руководителями одного производственного подразделения с рядовыми сотрудниками другого;
	\item бытовые взаимоотношения — складываются вне трудовой деятельности на отдыхе и в быту;
	\item формальные (официальные) отношения — нормативно предусмотренные взаимоотношения, закрепленные в официальных документах;
	\item неформальные (неофициальные) отношения — взаимоотношения, которые реально складываются при взаимоотношениях между людьми и проявляются в предпочтениях, симпатиях или антипатиях, взаимных оценках, авторитете и т.д.
\end{enumerate}

На характер межличностных отношений оказывают влияние такие личностные особенности, как пол, национальность, возраст, темперамент, состояние здоровья, профессия, опыт общения с людьми, самооценка, потребность в общении и др.

Этапы развития межличностных отношений:
\begin{enumerate}
	\item этап знакомства — первый этап — возникновение взаимного контакта, взаимного восприятия и оценки людьми друг друга, что во многом обусловливает и характер взаимоотношений между ними;
	\item этап приятельских отношений — возникновение межличностных отношений, формирование внутреннего отношения людей друг к другу на рациональном (осознание взаимодействующими людьми достоинств и недостатков друг друга) и эмоциональном уровнях (возникновение соответствующих переживаний, эмоционального отклика и т.д.);
	\item товарищеские отношения — сближение взглядов и оказание поддержки друг другу; характеризуются доверием.
\end{enumerate}
\chapter{Общение}
\section{Сущность общения}
Межличностное общение выступает необходимым условием бытия людей, без которого невозможно полноценное формирование не только отдельных психических функций, процессов и свойств человека, но и личности в целом. Вот почему изучение этого сложнейшего психического феномена как системного образования, имеющего многоуровневую структуру и только ему присущие характеристики, является актуальным для психологической науки. \cite{5}

Сущность межличностного общения заключается во взаимодействии человека с человеком. Именно этим оно отличается от других видов деятельности, когда происходит взаимодействие человека с каким-либо предметом или вещью.

Взаимодействующие при этом личности удовлетворяют свою потребность в общении друг с другом, в обмене информацией и пр. Например, обсуждение двумя прохожими конфликтной ситуации, свидетелями которой они только что оказались, или общение при знакомстве молодых людей друг с другом.

В подавляющем большинстве случаев межличностное общение почти всегда оказывается вплетенным в ту или иную деятельность и выступает как условие ее выполнения.

Межличностное общение является не только необходимым компонентом деятельности людей, осуществление которой предполагает их сотрудничество, но и обязательным условием нормального функционирования их общностей (например, школьного класса или производственной бригады рабочих). При сравнении характера межличностного общения в этих объединениях обращает на себя внимание как сходство, так и различие между ними.

Сходство заключается в том, что общение в них выступает необходимым условием бытия, этих объединений, фактором, от которого зависит успешность решения стоящих перед ними задач.

На общение влияет не только главная для данной общности деятельность, но и то, что представляет из себя сама эта общность. Например, если это школьный класс, то важно знать, насколько он сформирован как коллектив, какие оценочные эталоны в нем господствуют, если бригада — то каковы степень развития трудовой активности, уровень производственной квалификации каждого работника и т. д.

Особенности межличностного взаимодействия в любой общности в значительной степени обусловлены тем, как ее члены воспринимают и понимают друг друга, какой эмоциональный отклик преимущественно вызывают друг в друге и какой стиль поведения избирают.

Общности, к которым принадлежит человек, формируют эталоны общения, задают образцы поведения, которым личность приучается следовать повседневно при взаимодействии с другими людьми. Эти общности прямо воздействуют на развитие у него оценок, которые определяют восприятие им других людей, отношения и стиль общения с ними. Причем воздействие тем сильнее, чем авторитетнее общность в глазах человека.

Взаимодействуя с другими людьми, человек может одновременно выступать в роли и субъекта, и объекта общения. Как субъект он познает своего партнера, определяет к нему свое отношение (это может быть интерес, безразличие или неприязнь), воздействует на него с целью решения какой-либо конкретной задачи. В свою очередь, он сам является объектом познания для того, с кем общается. Партнер адресует ему свои чувства и старается на него повлиять. При этом следует подчеркнуть, что пребывание человека одновременно в двух «ипостасях» — объекта и субъекта — характерно для любого вида непосредственного общения людей, будь то общение одного учащегося с другим или учащегося и учителя.

Общение, будучи одним из основных видов деятельности людей, не только постоянно выявляет существенные характеристики личности как объекта и субъекта общения, но и влияет на весь ход ее дальнейшего формирования, в первую очередь на такие блоки свойств, в которых выражается отношение человека к другим людям и к себе. В свою очередь, изменения, происходящие в людях под давлением развертывающегося общения, воздействуют в той или иной степени на такие базисные свойства личности, в которых проявляется ее отношение к различным социальным институтам и общностям людей, природе, общественной и личной собственности, к труду.

\section{Теоретические подходы к исследованию общения}
Информационные подходы базируются на трех основных положениях: \cite{9}
\begin{enumerate}
	\item содержание информации может быть преобразовано в различные символы;
	\item человек является своеобразным экраном, на который «проецируется» передаваемая информация после ее восприятия и переработки;
	\item существует некое пространство, в котором взаимодействуют дискретные организмы и объекты ограниченного объема.
\end{enumerate}

В рамках информационного подхода было разработано две основные модели:
\begin{enumerate}
	\item модель К. Шеннона и В. Вивера, представляющая изменения сообщений в различные изображения, знаки, сигналы, символы, языки или коды и их последующую декодировку. Модель включала пять элементов, организованных в линейном порядке: источник информации—передатчик информации (шифратор)— канал для передачи сигналов — приемник информации (дешифратор) — получатель информации. Позже она была дополнена такими понятиями, как «обратная связь» (отклик получателя информации), «шум» (искажения и помехи в сообщении при его прохождении по каналу), «фильтры» (преобразователи сообщения, когда оно достигает шифратора или покидает Дешифратор) и др. Основным недостатком этой модели явилась недооценка других подходов в изучении проблемы общения;
	\item модель коммуникационного обмена, которая включала:
		\subitem a) коммуникационные условия;
		\subitem б) коммуникационное поведение;
		\subitem в) коммуникационные ограничения выбора стратегии общения;
		\subitem г) критерии интерпретации, определяющие и направляющие способы восприятия и оценки людьми своего поведения по отношению друг к другу.
\end{enumerate}

Интеракциональные подходы—рассматривают общение как ситуацию совместного присутствия, которая взаимно устанавливается и поддерживается людьми при помощи различных форм поведения и внешних атрибутов (внешности, предметов, обстановки и т.п.).В рамках интеракционных подходов было разработано пять моделей организации общения:
\begin{enumerate}
	\item лингвистическая модель, согласно которой все взаимодействия образуются и комбинируются из 50-60 элементарных движений и поз тела человека, а поведенческие акты, сформированные из этих единиц, организуются по принципу организации звуков в словах;
	\item модель социального навыка основывается на идее научения общению в самом общении;
	\item равновесная модель предполагает, что любое изменение поведения обычно компенсируется другим изменением, и наоборот (например, диалог — монолог, сочетание вопросов и ответов);
	\item программная модель социального взаимодействия предполагает, что общая структура межличностного взаимодействия порождается благодаря действию по крайней мере трех видов программ:
		\subitem a) программы, имеющей дело с простой координацией движений;
		\subitem б) программы, контролирующей изменение видов активности индивидов в ситуации, когда возникают помехи или неопределенности;
		\subitem в) программы, управляющей комплексной задачей мета-общения.
		
	Эти программы усваиваются индивидами по мере научения и позволяют организовывать разнородный поведенческий материал. Они «запускаются» в зависимости от содержательного контекста конкретной ситуации, задачи и социальной организации;
	\item системная модель рассматривает взаимодействие как конфигурацию систем поведения, управляющих обменом речевых высказываний и использованием пространства и территории взаимодействия.
\end{enumerate}

Реляционный подход строится на том, что общение — это система взаимоотношений, которые люди развивают друг с другом, с обществом и средой обитания, в которой они живут. Под информацией же понимается любое изменение какой-либо части этой системы, вызывающее изменение других частей. Люди, животные или другие организмы являются неотъемлемой частью процесса общения с момента рождения до момента смерти.

\section{Структура общения}
В структуре общения различают: \cite{28}
\begin{enumerate}
	\item коммуникативную сторону;
	\item интерактивную сторону;
	\item перцептивную сторону.
\end{enumerate}

Коммуникативная сторона общения выражается в обмене информацией между людьми. 

Особенности процесса обмена информацией в процессе человеческого общения:
\begin{enumerate}
	\item происходит не только передача информации, но и ее формирование, уточнение и развитие;
	\item обмен информацией сочетается с отношением людей друг к другу;
	\item происходит взаимное влияние и воздействие людей друг на друга;
	\item  коммуникативное влияние людей друг на друга возможно только при совпадении систем кодификации у коммуникатора (отправителя) и реципиента (принимающего);
	\item возможно возникновение специфических коммуникативных барьеров социального и психологического характера.
\end{enumerate}

Структурные компоненты общения как коммуникативной деятельности:
\begin{enumerate}
	\item субъект общения — коммуникатор;
	\item объект общения — реципиент;
	\item предмет общения — содержательная часть отправляемой информации;
	\item действия общения—единицы коммуникативной активности;
	\item средства общения — операции, с помощью которых осуществляются действия общения;
	\item продукт общения — образование материального и духовного характера как итог общения.
\end{enumerate}

Интерактивная сторона общения проявляется во взаимодействии людей друг с другом, т.е. обмене информацией, побуждениями, действиями. Цель взаимодействия состоит в удовлетворении своих потребностей, интересов, реализации целей, планов, намерений.

Типы взаимодействия:
\begin{enumerate}
	\item положительные—взаимодействия, направленные на организацию совместной деятельности: кооперация; согласие; приспособление; ассоциация; 
	\item отрицательные — взаимодействия, направленные на нарушение совместной деятельности, создание для нее препятствий: конкуренция; конфликт; оппозиция; диссоциация.
\end{enumerate}

Факторы, влияющие на тип взаимодействия:
\begin{enumerate}
	\item степень единства подходов к решению проблем;
	\item понимание обязанностей и прав;
	\item способы решения возникающих проблем и др.
\end{enumerate}

Перцептивная сторона общения выражается в процессе восприятия, изучения и оценки партнерами друг друга.

Структурные элементы социальной перцепции:
\begin{enumerate}
	\item субъект межличностного восприятия — тот, кто воспринимает (изучает) в процессе общения;
	\item объект восприятия — тот, кого воспринимают (познают) в процессе общения;
	\item процесс познания — включает познание, обратную связь, элементы коммуникации.
\end{enumerate}

В процессе общения человек выступает сразу в двух ипостасях: как объект и как субъект познания.

Факторы, влияющие на процесс межличностного восприятия:
\begin{enumerate}
	\item особенности субъекта: половые различия (женщины точнее идентифицируют эмоциональные состояния, достоинства и недостатки личности, мужчины—уровень интеллекта); возраст, темперамент (экстраверты точнее воспринимают, интроверты — оценивают); социальный интеллект (чем выше уровень социальных и общих знаний, тем точнее оценка при восприятии); психическое состояние; состояние здоровья; установки—предшествующая оценка объектов восприятия; ценностные ориентации; уровень социально-психологической компетентности и т.д.
	\item особенности объекта: физический облик (антропологические — рост, телосложение, цвет кожи и т.д., физиологические — дыхание, кровообращение, функциональные — осанка, поза и походка и паралингвистические — мимика, жесты и телодвижения); социальный облик: социальная роль, внешний облик, проксемические особенности общения (расстояние и расположение общающихся), речевые и экстралингвистические характеристики (семантика, грамматика и фонетика), деятельностные особенности;
	\item отношения между субъектом и объектом восприятия;
	\item ситуация, в которой происходит перцепция.
\end{enumerate}

\section{Виды общения}
Виды общения по средствам: \cite{13}
\begin{enumerate}
	\item вербальное общение — осуществляется посредством речи и является прерогативой человека. Оно предоставляет человеку широкие коммуникативные возможности и гораздо богаче всех видов и форм невербального общения, хотя в жизни не может полностью его заменить;
	\item невербальное общение происходит с помощьюмимики, жестов и пантомимики, через прямые сенсорные или телесные контакты (тактильные, зрительные, слуховые, обонятельные и другие ощущения и образы, получаемые от другого лица). Невербальные формы и средства общения присущи не только человеку, но и некоторым животным (собакам, обезьянам и дельфинам). В большинстве случаев невербальные формы и средства общения человека являются врожденными. Они позволяют людям взаимодействовать друг с другом, добиваясь взаимопонимания на эмоциональном и поведенческом уровнях. Важнейшей невербальной составляющей процесса общения является умение слушать.
\end{enumerate}
 
Виды общения по целям:
\begin{enumerate}
	\item биологическое общение связано с удовлетворением основных органических потребностей и необходимо для поддержания, сохранения и развития организма;
	\item социальное общение направлено на расширение и укрепление межличностных контактов, установление и развитие интерперсональных отношений, личностного роста индивида.
\end{enumerate}

Виды общения по содержанию:
\begin{enumerate}
	\item материальное — обмен предметами и продуктами деятельности, которые служат средством удовлетворения их актуальных потребностей;
	\item когнитивное — передача информации, расширяющей кругозор, совершенствующей и развивающей способности;
	\item кондиционное — обмен психическими или физиологическими состояниями, оказание влияния друг на друга, рассчитанное на то, чтобы привести человека в определенное физическое или психическое состояние;
	\item деятельностное — обмен действиями, операциями, умениями, навыками;
	\item мотивационное общение состоит в передаче друг другу определенных побуждений, установок или готовности к действиям в определенном направлении. 
\end{enumerate}

По опосредованности:
\begin{enumerate}
	\item непосредственное общение — происходит с помощью естественных органов, данных живому существу природой: руки, голова, туловище, голосовые связки и т.п.;
	\item опосредствованное общение — связано с использованием специальных средств и орудий для организации общения и обмена информацией (природных(палка, брошенный камень, след на земле и т. д.) или культурных предметов (знаковые системы, записи символов на различных носителях, печать, радио, телевидение и т. п.));
	\item прямое общение строится на основе личных контактов и непосредственного восприятия друг другом общающихся людей в самом акте общения (например, телесные контакты, беседы людей и т.д.);
	\item косвенное общение происходит через посредников, которыми могут быть другие люди (например, переговоры между конфликтующими сторонами на межгосударственном, межнациональном, групповом, семейном уровнях).
\end{enumerate}

Другие виды общения:
\begin{enumerate}
	\item деловое общение — общение, целью которого является достижение какого-либо четкого соглашения или договоренности;
	\item воспитательное общение — предполагает целенаправленное воздействие одного участника на другого с достаточно четким представлением желаемого результата;
	\item диагностическое общение — общение, целью которого является формулировка определенного представления о собеседнике или получение от него какой-либо информации (таково общение врача с пациентом и т.п.);
	\item интимно-личностное общение — возможно при заинтересованности партнеров в установлении и поддержании доверительного и глубокого контакта, возникает между близкими людьми и в значительной степени является результатом предшествующих взаимоотношений.
\end{enumerate}

\section{Формы общения}
\begin{enumerate}
	\item монологическая — когда только одному из партнеров отводится роль активного участника, а другому — пассивного исполнителя (например, лекция, нотация и т.д.);
	\item диалоговая — характерно сотрудничество участников — собеседников или партнеров по общению (например, беседа, разговор);
	\item полилогическая — многостороннее общение, которое носит характер борьбы за коммуникативную инициативу.
\end{enumerate}

\section{Уровни общения}
В зарубежной и отечественной психологии имеются различные взгляды на уровни общения. \cite{28}

Уровни общения по Б.Г. Ананьеву:
\begin{enumerate}
	\item микроуровень—состоит из самых мелких элементов межличностногообщения с ближайшим окружением, с которым человек живет и чаще всего вступает в контакт (семья, друзья);
	\item мезоуровень — общение на уровне школы, производственной бригады и т.д.;
	\item макроуровень — включает такие крупные структуры, как управление и торговля. 
\end{enumerate}

Уровни общения по Э. Берну:
\begin{enumerate}
	\item ритуалы — это определенный порядок действий, которым совершается и закрепляется обычай;
	\item времяпрепровождение (просмотр телевизора, чтение книг, танцы и т.д.);
	\item игры—виды деятельности, результатом которых не становится производство какого-либо продукта;
	\item близость — интимные отношения;
	\item деятельность—специфический вид активности человека, направленный на познание и преобразование окружающего мира. 
\end{enumerate}

Наиболее распространенной в отечественной психологии является следующая уровневая система:
\begin{enumerate}
	\item примитивный уровень — предполагает реализацию схемы общения, в которой собеседник не партнер, а нужный или мешающий предмет. При этом фазы контакта исполняются в пристройке сверху или (с откровенно сильным партнером) снизу. Подобный уровень общения предлагается в состоянии опьянения, озлобления, в состоянии конфликта и т.д.;
	\item манипулятивный уровень—реализуется схема «партнер — соперник» в игре, которую непременно надо выиграть, причем выигрыш — выгода (материальная, житейская или психологическая). При этом манипулятор улавливает и пытается использовать слабые места партнера;
	\item стандартизованный уровень — общение, основанное на стандартах, когда один из партнеров (или оба) не желают контакта, но без него не обойтись;
	\item конвенциональный уровень — уровень обычного равноправного человеческого общения в рамках принятых правил поведения. Этот уровень требует от партнеров высокой культуры общения, которое может рассматриваться как искусство и для овладения которым иному человеку приходится годами работать над собой. Он является оптимальным для разрешения личных и межличностных проблем в человеческих контактах;
	\item игровой уровень — характеризуется так же, как конвенциональный, но с повышенной положительной направленностью на партнера, интересом к нему и желанием породить Подобный же интерес к себе со стороны партнера. В игре главное — заинтриговать, заинтересовать партнера. На этом уровне больше ценится возникшая человеческая связь, а не информативная компонента общения. Идеален для преподавательской деятельности;
	\item уровень делового общения — по сравнению с конвенциональным уровнем предполагает повышенную направленность на партнера как на участника коллективной деятельности. Главным на этом уровне является степень умственной и деловой активности партнера, его включенность в общую задачу. Идеален для групповой деятельности, для мозговых штурмов и т.д.;
	\item духовный уровень — высший уровень человеческого общения, для которого характерно взаиморастворение в партнере, высокая спонтанность мысли и чувства, предельная свобода самовыражения; партнер воспринимается как носитель духовного начала, и это начало пробуждает в нас чувство, которое сродни благоговению. 
\end{enumerate}

\section{Функции и средства общения}
Функции общения — это роли и задачи, которые выполняет общение в процессе социального бытия человека:  \cite{28}
\begin{enumerate}
	\item информационно-коммуникативная функция состоит в обмене информацией между индивидами. Составными элементами общения являются: коммуникатор (передает информацию), содержание сообщения, реципиент (принимает сообщение). Эффективность передачи информации проявляется в понимании информации, ее принятии или непринятии, усвоении. Для осуществления информационно-коммуникативной функции необходимо наличие единой или сходной системы кодификации/декодификации сообщений. Передача любой информации возможна посредством различных знаковых систем;
	\item побудительная функция—стимуляция активности партнеров для организации совместных действий;
	\item интегративная функция — функция объединения людей;
	\item функция социализации — общение способствуетвыработке навыков взаимодействия человека в обществе по принятым в нем нормам и правилам;
	\item координационная функция — согласование действий при осуществлении совместной деятельности;
	\item функция понимания—адекватное восприятие и понимание информации;
	\item регуляционно-коммуникативиая (интерактивная) функция общения направлена на регуляцию и коррекцию поведения при непосредственной организации совместной деятельности людей в процессе их взаимодействия;
	\item аффективно-коммуникативная функция общения состоит в воздействии на эмоциональную сферу человека, которое может быть целенаправленным или непроизвольным. 
\end{enumerate}

Средства общения — способы кодирования, передачи, переработки и расшифровки информации, передаваемой в процессе общения. Они бывают вербальные и невербальные. Вербальные средства общения—слова с закрепленными за ними значениями. Слова могут быть произнесены вслух (устная речь), написаны (письменная речь), заменены жестами у слепых или произнесены про себя. Устная речь является более простой и экономичной формой вербальных средств.

Она разделяется на:
\begin{enumerate}
	\item диалогическую речь, в которой принимают участие два собеседника;
	\item монологическую речь — речь, которую произносит один человек.
\end{enumerate}

Письменная речь применяется при невозможности устного общения или когда необходима точность, выверенность каждого слова.

Невербальные средства общения — знаковая система, которая дополняет и усиливает вербальную коммуникацию, а иногда и заменяет ее. С помощью невербальных средств общения передается около 55-65\% информации.

К невербальным средствам общения относятся:
\begin{enumerate}
	\item визуальные средства:
		\subitem a) кинестетические средства—это зрительно воспринимаемые движения другого человека, выполняющие выразительно-регулятивную функцию в общении. К кинесике относятся выразительные движения, проявляющиеся в мимике, позе, жесте, взгляде, походке;
		\subitem б) направление взгляда и визуальный контакт;
		\subitem в) выражение лица;
		\subitem г) выражение глаз;
		\subitem д) поза — расположение тела в пространстве («нога на ногу», перекрещенные руки, ноги и т.д.);
		\subitem е) дистанция (расстояние до собеседника, угол поворота к нему, персональное пространство);
		\subitem ж) кожные реакции (покраснение, испарина);
		\subitem з) вспомогательные средства общения (особенности телосложения (половые, возрастные)) и средства их преобразования (одежда, косметика, очки, украшения, татуировка, усы, борода, сигарета и т. п.);
	\item акустические (звуковые):
		\subitem a) связанные с речью (громкость, тембр, интонация, тон, высота звука, ритм, речевые паузы и их локализация в тексте);
		\subitem б) не связанные с речью (смех, скрежет зубов, плач, кашель, вздохи и т.п.);
	\item тактильные — связанные с прикосновением:
		\subitem a) физическое воздействие (ведение слепого за руку и др.);
		\subitem б) такевика (пожатие руки, хлопание по плечу). 
\end{enumerate}

\chapter{Взаимосвязь общения и отношения}
В психологической науке выполняется очень много исследований, в которых тот или иной более простой или более сложный феномен освещается сам по себе, не в связи с другими явлениями, а это всегда обедняет значение получаемых результатов, потому что по-настоящему понять сущность любого явления можно, лишь постигая его во взаимодействии с другими явлениями. \cite{17}

Сказанное полностью приложимо к состоянию изучения такого сложного психологического феномена, каким является общение, а также такого личностного образования, как отношение. 

Когда говорят об общении, то обычно имеют в виду взаимодействие между людьми, осуществляемое с помощью средств речевого и неречевого воздействия и преследующее цель достижения изменений в познавательной, мотивационно-эмоциональной и поведенческой сферах участвующих в общении лиц. Под отношением же, как известно, понимается психологический феномен, сутью которого является возникновение у человека психического образования, аккумулирующего в себе результаты познания конкретного объекта действительности (в общении это другой человекили общность людей), интеграции всех состоявшихся эмоциональных откликов на этот объект, а также поведенческих ответов на него. 

Самой важной психической составляющей отношения оказывается мотивационно-эмоциональный компонент, который сигнализирует о валентности отношения — положительной, отрицательной, противоречивой или безразличной.

Когда один человек вступает в общение с другим, то оба они фиксируют особенности внешнего облика друг друга, «прочитывают» переживаемые состояния, воспринимают и истолковывают тем или иным образом поведение, так или иначе расшифровывают цели и мотивы этого поведения. И внешний облик, и состояние, и поведение, и приписываемые человеку цели и мотивы всегда вызывают у общающейся с ним личности какое-то отношение, причем оно может дифференцироваться по своему характеру и силе в зависимости от того, какая сторона в другом человеке его вызвала. 

Особой проблемой при изучении взаимозависимостей общения и отношения является установление соответствия характера и способов выражения отношения. Формируясь как личности в конкретной социальной среде, люди усваивают и характерный для этой среды язык выражения отношений. Не говоря сейчас об особенностях выражения отношений, отмечаемых у представителей различных этнических общностей, важно иметь в виду, что даже в границах одной этнической общности, но в ее разных социальных группах названный язык может иметь свою весьма определенную специфику.

Формой выражения отношения могут стать и действие, и поступок.

Межличностное общение отличается от общения межролевого, что участники такого общения стараются, решая свои задачи, делать поправку при выборе поведения, передающего отношение, на индивидуально неповторимые особенности друг друга. Уместно добавить, что умение психологически искусно инструментовать форму выражения своих отношений крайне необходимо лицам, основной деятельностью которых является воспитание и детей, и молодежи, и взрослых.

Обсуждая проблему взаимосвязи общения и отношения, а также зависимости между содержанием отношения и формой его выражения, следует подчеркнуть, что выбор человеком наиболее психологически целесообразной формы выражения своего отношения в общении происходит без напряжения и бросающейся в глаза нарочитости, если у него сформированы психические свойства личности, которые обязательны для успешного межличностного общения. Это прежде всего способность к идентификации и децентрации, эмпатии и саморефлексии.

Для действительной полноты анализа общения и связей его с отношениями необходимо оценивать, по крайней мере, главные объективные и субъективные характеристики этого процесса, имея также в виду и одного, и другого взаимодействующих в нем людей (если это диадное общение). Эти прослеженные в самом первом приближении связи разных характеристик общения и отношения показывают, насколько велико их значение в субъективном мире каждого человека, насколько весома их роль в детерминировании психического самочувствия человека, в определении картины его поведения. Поэтому чрезвычайно важно развернуть систематические исследования на теоретическом, экспериментальном и прикладном уровнях всех наиболее существенных аспектов взаимозависимостей общения и отношения. Планируя эти исследования, надо ясно видеть, что в изучении взаимосвязей общения и отношения должны принять участие все основные области психологической науки и обязательно педагоги, занимающиеся разработкой теории и методического инструментария воспитания.

%\chapter{Технологическая часть}
%\chapter{Экспериментальная часть}


\backmatter %% Здесь заканчивается нумерованная часть документа и начинаются ссылки и
            
\StructuredChapter{Заключение}
Рассматривая межличностные отношения можно сделать вывод, что межличностные отношения – субъективно переживаемые связи между людьми, объективно проявляющиеся в характере и способах межличностного взаимодействия, т.е. взаимных влияний, оказываемых людьми друг на друга в процессе их совместной деятельности и общения. 

Межличностные отношения – это система установок, ориентаций и ожиданий членов группы относительно друг друга, определяющихся содержанием и организацией совместной деятельности и ценностями, на которых основывается общение людей. при этом возможно рассогласование субъективно переживаемых и объективно существующих связей индивида с другими людьми. В группах разного уровня развития. 

Межличностные отношения различаются не только количественно, но и качественно. Так, в коллективе они составляют сложную иерархическую структуру, которая развивается по мере включения его в общественно значимую деятельность. Экспериментальное исследование межличностных отношений осуществляется социальной психологией с помощью специальных методик: Социометрия, Референтометрический метод, Методы исследования личности. Чаще всего в практике используется Социометрической метод Дж. Морено. 

Общение можно охарактеризовать как сложный, многоплановый процесс установления и развития контактов между людьми, порождаемый потребностями совместной деятельности и включающий в себя обмен информацией, выработку единой стратегии взаимодействия, восприятие и понимание другого человека. Соответственно в общении различаются три стороны: коммуникативная, интерактивная и перцептивная. Где коммуникативная сторона общения связана с выявлением информационного процесса между людьми как активными субъектами, т.е. с учетом отношений между партнерами, их установок, целей, намерений, что приводит не просто к «движению» информации, но к уточнению и обогащению тех знаний, сведений, мнений, которыми обмениваются люди. Средствами коммуникативного процесса являются различные знаковые системы, прежде всего речь, а также оптико-кинетическая система знаков (жесты, мимика, пантомимика), пара- и экстралингвистическая системы (интонация, неречевые вкрапления в речь, например, паузы), система организации пространства и времени коммуникации, система «контакта глазами». Интерактивная сторона общения представляет собой построение общей стратегии взаимодействия. Различают ряд типов взаимодействия между людьми, прежде всего кооперацию и конкуренцию. Перцептивная сторона общения включает в себя процесс формирования образа другого человека, что достигается «прочтением» за физическими характеристиками человека, его психологических свойств и особенностей его поведения. Основными механизмами познания другого человека являются идентификация и рефлексия. 

Самой важной психической составляющей отношения оказывается мотивационно-эмоциональный компонент, который сигнализирует о валентности отношения – положительной, отрицательной, противоречивой или безразличной. 

Особой проблемой при изучении взаимозависимостей общения и отношения является установление соответствия характера и способов выражения отношения; так же влияет социальное значение и система ценностей. 
 %% заключение


\addcontentsline{toc}{chapter}{Список использованных источников}
\UnnamedStructuredChapter{Список использованных источников}
\bibliographystyle{utf8gost705u}
\begin{thebibliography}{1}
%	\bibitem{1}
%	Андреева Г.М. Социальная психология. Учебник для высших учебных заведений / Г.М. Андреева. - М.: Аспект Пресс, 2008. - 378 с. 
	
%	\bibitem{2}
%	Андриенко Е.В. Социальная психология: учебное пособие для студентов пед.вузов. М.: 2007.
	
	\bibitem{3}
	Аскевис-Леерпе, Ф. Психология: краткий курс/Ф. Аскевис-Леерпе, К. Барух, А. Картрон; пер. с франц. М.Л. Карачун. - М.: АСТ: Астрель, 2006. - 155 с.
	
	\bibitem{4}
	Бодалев А.А. Психология межличностного общения. Рязань, 1994.
	
	\bibitem{28}
	Челдышова Н. Б. Шпаргалка по социальной психологии. М.: Экзамен, 2009.
	
	\bibitem{5}
	Бодалев А.А. Психология общения. Избранные психологические труды. - 3-е изд., перераб. и допол. - М.: Издательство Московского психолого социального института; Воронеж: Издательство НПО "МОДЭК", 2002.- 320с.
	
%	\bibitem{6}
%	Большая энциклопедия психологических тестов. М.: Изд-во Эксмо, 2005. - 416 с.
	
%	\bibitem{7}
%	Вердербер, Р., Вердербер, К. Психология общения. - СПб.: прайм - ЕВРОЗНАК, 2003. - 320 с.
	
%	\bibitem{8}
%	Выготский Л.С. Психология развития человека. М.: ЭКСМО, 2003.
	
	\bibitem{9}
	Глейтман Г. Фридлунд А., Райсберг Д. Основы психологии. Спб.: Речь, 2001.
	
%	\bibitem{10}
%	Горянина В.А. Психология общения: Учеб.пособие для студ. Высш. Учеб. Заведений. - М.:Издательский центр "Академия", 2002. - 416 с.
	
%	\bibitem{11}
%	Дружинин В.Н. Структура и логика психологического исследования. М.: ИП РАН, 1994.
	
%	\bibitem{12}
%	Ермолаев О.Ю. Математическая статистика для психологов: Учебник /О.Ю. Ермолаев. – 2-е изд., испр. – М.: Московский психолого-социальный институт: Флинта, 2003. – 336 с. 
	
	\bibitem{13}
	Емельянов Ю.Н., Кузьмин Е.С. Теоретические и методические основы социально-психологического тренинга. Л.: ЛГУ, 1983. - 103 с.
	
%	\bibitem{14}
%	Краткий психологический словарь /Сост. Л.А. Карпенко; Под. Общ. ред. А.В . Петровского, М.Г. Ярошевского. - М.: Политиздат, 1985. - 431 с. 
	
%	\bibitem{15}
%	Крысько В.Г. Социальная психология: словарь-справочник. - Мн.: Харвест, 2004. - 688 с.	
	
%	\bibitem{16}
%	Крысько В.Г. Социальная психология: Учеб. для вузов. 2-е изд. - СПб.: Питер, 2006. - 432 с. 
	
	\bibitem{17}
	Леонтьев А.Н. Деятельность, сознание, личность. М.: Смысл: Издательский центр «Академия», 2006. 
	
%	\bibitem{18}
%	Мокшанцев Р.И., Мокшанцева А.В. Социальная психология: учеб. Пособие для вузов. М.: 2001.
	
%	\bibitem{19}
%	Прутченков А.С. Социально-психологический тренинг межличностного общения. М., 1991 - 45 с. 
	
%	\bibitem{20}
%	Психологические тесты /Под ред. А.А. Карелина: В 2 т. - М.: Гуманит. изд. центр ВЛАДОС, 2003. - Т.2. - 248 с. 
	
%	\bibitem{21}
%	Психология и педагогика военного управления. Учебно-методическое пособие. /Изд. ВВИА им. В.В. Жуковского, 1992. 
	
%	\bibitem{22}
%	Ребер А. Большой толковый психологический словарь. В 2-х томах. М.: Вече, АСТ, 2000. 
	
%	\bibitem{23}
%	Семечкин, Н.И. Социальная психология: учебник для вузов. - СПб.: Питер, 2004. - 376 с.
	
%	\bibitem{24}
%	Социальная психология: учебное пособие для вузов /Под ред. А.А. Журавлева. М.: 2003.
	
%	\bibitem{25}
%	Справочник практического психолога. Психодиагностика/под.общ.ред. С.Т.Посоховой. - М.: АСТ; СПб.: Сова, 2005. - 671, [1] с.:ил.
	
%	\bibitem{26}
%	Фолкэн Чак Т. Психология - это просто / Пер. с англ. Р.Муртазина. - М.: ФАИР-ПРЕСС, 2001. - 640 с. 
	
%	\bibitem{27}
%	Шевандрин Н.И. Социальная психология в образовании. М. 1995.
	
\end{thebibliography}



\appendix   % Тут идут приложения

%\include{90-appendix1}

%\include{91-appendix2}

\end{document}

%%% Local Variables:
%%% mode: latex
%%% TeX-master: t
%%% End:
