\chapter{Межличностные отношения}
\section{Место и природа межличностных отношений}
В социально-психологической литературе высказываются различные точки зрения на вопрос о том, где «расположены» межличностные отношения, прежде всего относительно системы общественных отношений. Природа межличностных отношений может быть правильно понята, если их не ставить в один ряд с общественными отношениями, а увидеть в них особый ряд отношений, возникающий внутрикаждого вида общественных отношений, не вне их. 

Природа межличностных отношений существенно отличается от природы общественных отношений: их важнейшая специфическая черта — эмоциональная основа. Поэтому межличностные отношения можно рассматривать как фактор психологического «климата» группы. Эмоциональная основа межличностных отношений означает, что они возникают и складываются на основе определенных чувств, рождающихся у людей по отношению друг к другу. В отечественной школе психологии различаются три вида, или уровня эмоциональных проявлений личности: аффекты, эмоции и чувства. Эмоциональная основа межличностных отношений включает все виды этих эмоциональных проявлений. \cite{4}

Отношения между людьми не складываются лишь на основе непосредственных эмоциональных контактов. Сама деятельность задает и другой ряд отношений, опосредованных ею. Поэтому-то и является чрезвычайно важной и трудной задачей социальной психологии одновременный анализ двух рядов отношений в группе: как межличностных, так и опосредованных совместной деятельностью, т.е. в конечном счете стоящих за ними общественных отношений.

Все это ставит очень остро вопрос о методических средствах такого анализа. Традиционная социальная психология обращала преимущественно свое внимание на межличностные отношения, поэтому относительно их изучения значительно раньше и полнее был разработан арсенал методических средств. Главным из таких средств является широко известный в социальной психологии метод социометрии, предложенный американским исследователем Дж. Морено, для которого она есть приложение к его особой теоретической позиции. Хотя несостоятельность этой концепции давно подвергнута критике, методика, разработанная в рамках этой теоретической схемы, оказалась весьма популярной.

Таким образом, мы можем сказать, что межличностные отношения рассматриваются как фактор психологического «климата» группы. Но для диагностики межличностных и межгрупповых отношений в целях их изменения, улучшения и совершенствования применяется социометрическая техника, основоположником которой является американский психиатр и социальный психолог Дж. Морено. 

\section{Сущность межличностных отношений}
Межличностные отношения — это совокупность связей, складывающихся между людьми в форме чувств, суждений и обращений друг к другу. \cite{28}

Межличностные отношения включают:
\begin{enumerate}
	\item восприятие и понимание людьми друг друга;
	\item межличностную привлекательность (притяжение и симпатия);
	\item взаимодействие и поведение (в частности, ролевое).
\end{enumerate}

Компоненты межличностных отношений:

\begin{enumerate}
	\item когнитивный компонент — включает в себя все познавательные психические процессы: ощущения, восприятие, представление, память, мышление, воображение. Благодаря этому компоненту происходит познание индивидуально-психологических особенностей партнеров по совместной деятельности и взаимопонимание между людьми. Характеристиками взаимопонимания являются:
		\subitem a) адекватность — точность психического отражения воспринимаемой личности;
		\subitem б) идентификация — отождествление индивидом своей личности с личностью другого индивида;
	\item эмоциональный компонент — включает положительные или отрицательные переживания, возникающие у человека при межличностном общении с другими людьми:
		\subitem a) симпатии или антипатии;
		\subitem б) удовлетворенность собой, партнером, работой и т.д.;
		\subitem в) эмпатия — эмоциональный отклик на переживания другого человека, который может проявляться в виде сопереживания (переживания тех чувств, которые испытывает другой), сочувствия (личностного отношения к переживаниям другого) и соучастия (сопереживание, сопровождаемое содействием);
	\item поведенческий компонент — включает мимику, жестикуляцию, пантомимику, речь и действия, выражающие отношения данного человека к другим людям, к группе в целом. Он играет ведущую роль в регулировании взаимоотношений. Эффективность межличностных отношений оценивается по состоянию удовлетворенности— неудовлетворенности группы и ее членов.
\end{enumerate}

Виды межличностных отношений:
\begin{enumerate}
	\item производственные отношения — складываются между сотрудниками организаций при решении производственных, учебных, хозяйственных, бытовых и др. проблем и предполагают закрепленные правила поведения сотрудников по отношению друг к другу. Разделяются на отношения:
		\subitem a) по вертикали — между руководителями и подчиненными;
		\subitem б) по горизонтали — отношения между сотрудниками, имеющими одинаковый статус;
		\subitem б) по диагонали — отношения между руководителями одного производственного подразделения с рядовыми сотрудниками другого;
	\item бытовые взаимоотношения — складываются вне трудовой деятельности на отдыхе и в быту;
	\item формальные (официальные) отношения — нормативно предусмотренные взаимоотношения, закрепленные в официальных документах;
	\item неформальные (неофициальные) отношения — взаимоотношения, которые реально складываются при взаимоотношениях между людьми и проявляются в предпочтениях, симпатиях или антипатиях, взаимных оценках, авторитете и т.д.
\end{enumerate}

На характер межличностных отношений оказывают влияние такие личностные особенности, как пол, национальность, возраст, темперамент, состояние здоровья, профессия, опыт общения с людьми, самооценка, потребность в общении и др.

Этапы развития межличностных отношений:
\begin{enumerate}
	\item этап знакомства — первый этап — возникновение взаимного контакта, взаимного восприятия и оценки людьми друг друга, что во многом обусловливает и характер взаимоотношений между ними;
	\item этап приятельских отношений — возникновение межличностных отношений, формирование внутреннего отношения людей друг к другу на рациональном (осознание взаимодействующими людьми достоинств и недостатков друг друга) и эмоциональном уровнях (возникновение соответствующих переживаний, эмоционального отклика и т.д.);
	\item товарищеские отношения — сближение взглядов и оказание поддержки друг другу; характеризуются доверием.
\end{enumerate}