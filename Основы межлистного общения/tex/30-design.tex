\chapter{Взаимосвязь общения и отношения}
В психологической науке выполняется очень много исследований, в которых тот или иной более простой или более сложный феномен освещается сам по себе, не в связи с другими явлениями, а это всегда обедняет значение получаемых результатов, потому что по-настоящему понять сущность любого явления можно, лишь постигая его во взаимодействии с другими явлениями \cite{17}. 

Сказанное полностью приложимо к состоянию изучения такого сложного психологического феномена, каким является общение, а также такого личностного образования, как отношение. 

Когда говорят об общении, то обычно имеют в виду взаимодействие между людьми, осуществляемое с помощью средств речевого и неречевого воздействия и преследующее цель достижения изменений в познавательной, мотивационно-эмоциональной и поведенческой сферах участвующих в общении лиц. Под отношением же, как известно, понимается психологический феномен, сутью которого является возникновение у человека психического образования, аккумулирующего в себе результаты познания конкретного объекта действительности (в общении это другой человекили общность людей), интеграции всех состоявшихся эмоциональных откликов на этот объект, а также поведенческих ответов на него. 

Самой важной психической составляющей отношения оказывается мотивационно-эмоциональный компонент, который сигнализирует о валентности отношения — положительной, отрицательной, противоречивой или безразличной.

Когда один человек вступает в общение с другим, то оба они фиксируют особенности внешнего облика друг друга, «прочитывают» переживаемые состояния, воспринимают и истолковывают тем или иным образом поведение, так или иначе расшифровывают цели и мотивы этого поведения. И внешний облик, и состояние, и поведение, и приписываемые человеку цели и мотивы всегда вызывают у общающейся с ним личности какое-то отношение, причем оно может дифференцироваться по своему характеру и силе в зависимости от того, какая сторона в другом человеке его вызвала. 

Особой проблемой при изучении взаимозависимостей общения и отношения является установление соответствия характера и способов выражения отношения. Формируясь как личности в конкретной социальной среде, люди усваивают и характерный для этой среды язык выражения отношений. Не говоря сейчас об особенностях выражения отношений, отмечаемых у представителей различных этнических общностей, важно иметь в виду, что даже в границах одной этнической общности, но в ее разных социальных группах названный язык может иметь свою весьма определенную специфику.

Формой выражения отношения могут стать и действие, и поступок.

Межличностное общение отличается от общения межролевого, что участники такого общения стараются, решая свои задачи, делать поправку при выборе поведения, передающего отношение, на индивидуально неповторимые особенности друг друга. Уместно добавить, что умение психологически искусно инструментовать форму выражения своих отношений крайне необходимо лицам, основной деятельностью которых является воспитание и детей, и молодежи, и взрослых.

Обсуждая проблему взаимосвязи общения и отношения, а также зависимости между содержанием отношения и формой его выражения, следует подчеркнуть, что выбор человеком наиболее психологически целесообразной формы выражения своего отношения в общении происходит без напряжения и бросающейся в глаза нарочитости, если у него сформированы психические свойства личности, которые обязательны для успешного межличностного общения. Это прежде всего способность к идентификации и децентрации, эмпатии и саморефлексии.

Для действительной полноты анализа общения и связей его с отношениями необходимо оценивать, по крайней мере, главные объективные и субъективные характеристики этого процесса, имея также в виду и одного, и другого взаимодействующих в нем людей (если это диадное общение). Эти прослеженные в самом первом приближении связи разных характеристик общения и отношения показывают, насколько велико их значение в субъективном мире каждого человека, насколько весома их роль в детерминировании психического самочувствия человека, в определении картины его поведения. Поэтому чрезвычайно важно развернуть систематические исследования на теоретическом, экспериментальном и прикладном уровнях всех наиболее существенных аспектов взаимозависимостей общения и отношения. Планируя эти исследования, надо ясно видеть, что в изучении взаимосвязей общения и отношения должны принять участие все основные области психологической науки и обязательно педагоги, занимающиеся разработкой теории и методического инструментария воспитания.
