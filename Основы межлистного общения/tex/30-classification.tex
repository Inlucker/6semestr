\chapter{Общение}
\section{Сущность общения}
Межличностное общение выступает необходимым условием бытия людей, без которого невозможно полноценное формирование не только отдельных психических функций, процессов и свойств человека, но и личности в целом. Вот почему изучение этого сложнейшего психического феномена как системного образования, имеющего многоуровневую структуру и только ему присущие характеристики, является актуальным для психологической науки \cite{5}. 

Сущность межличностного общения заключается во взаимодействии человека с человеком. Именно этим оно отличается от других видов деятельности, когда происходит взаимодействие человека с каким-либо предметом или вещью.

Взаимодействующие при этом личности удовлетворяют свою потребность в общении друг с другом, в обмене информацией и пр. Например, обсуждение двумя прохожими конфликтной ситуации, свидетелями которой они только что оказались, или общение при знакомстве молодых людей друг с другом.

В подавляющем большинстве случаев межличностное общение почти всегда оказывается вплетенным в ту или иную деятельность и выступает как условие ее выполнения.

Межличностное общение является не только необходимым компонентом деятельности людей, осуществление которой предполагает их сотрудничество, но и обязательным условием нормального функционирования их общностей (например, школьного класса или производственной бригады рабочих). При сравнении характера межличностного общения в этих объединениях обращает на себя внимание как сходство, так и различие между ними.

Сходство заключается в том, что общение в них выступает необходимым условием бытия, этих объединений, фактором, от которого зависит успешность решения стоящих перед ними задач.

На общение влияет не только главная для данной общности деятельность, но и то, что представляет из себя сама эта общность. Например, если это школьный класс, то важно знать, насколько он сформирован как коллектив, какие оценочные эталоны в нем господствуют, если бригада — то каковы степень развития трудовой активности, уровень производственной квалификации каждого работника и т. д.

Особенности межличностного взаимодействия в любой общности в значительной степени обусловлены тем, как ее члены воспринимают и понимают друг друга, какой эмоциональный отклик преимущественно вызывают друг в друге и какой стиль поведения избирают.

Общности, к которым принадлежит человек, формируют эталоны общения, задают образцы поведения, которым личность приучается следовать повседневно при взаимодействии с другими людьми. Эти общности прямо воздействуют на развитие у него оценок, которые определяют восприятие им других людей, отношения и стиль общения с ними. Причем воздействие тем сильнее, чем авторитетнее общность в глазах человека.

Взаимодействуя с другими людьми, человек может одновременно выступать в роли и субъекта, и объекта общения. Как субъект он познает своего партнера, определяет к нему свое отношение (это может быть интерес, безразличие или неприязнь), воздействует на него с целью решения какой-либо конкретной задачи. В свою очередь, он сам является объектом познания для того, с кем общается. Партнер адресует ему свои чувства и старается на него повлиять. При этом следует подчеркнуть, что пребывание человека одновременно в двух «ипостасях» — объекта и субъекта — характерно для любого вида непосредственного общения людей, будь то общение одного учащегося с другим или учащегося и учителя.

Общение, будучи одним из основных видов деятельности людей, не только постоянно выявляет существенные характеристики личности как объекта и субъекта общения, но и влияет на весь ход ее дальнейшего формирования, в первую очередь на такие блоки свойств, в которых выражается отношение человека к другим людям и к себе. В свою очередь, изменения, происходящие в людях под давлением развертывающегося общения, воздействуют в той или иной степени на такие базисные свойства личности, в которых проявляется ее отношение к различным социальным институтам и общностям людей, природе, общественной и личной собственности, к труду.

\section{Теоретические подходы к исследованию общения}
Информационные подходы базируются на трех основных положениях: 
\begin{enumerate}
	\item содержание информации может быть преобразовано в различные символы;
	\item человек является своеобразным экраном, на который «проецируется» передаваемая информация после ее восприятия и переработки;
	\item существует некое пространство, в котором взаимодействуют дискретные организмы и объекты ограниченного объема \cite{9}.
\end{enumerate}

В рамках информационного подхода было разработано две основные модели:
\begin{enumerate}
	\item модель К. Шеннона и В. Вивера, представляющая изменения сообщений в различные изображения, знаки, сигналы, символы, языки или коды и их последующую декодировку. Модель включала пять элементов, организованных в линейном порядке: источник информации—передатчик информации (шифратор)— канал для передачи сигналов — приемник информации (дешифратор) — получатель информации. Позже она была дополнена такими понятиями, как «обратная связь» (отклик получателя информации), «шум» (искажения и помехи в сообщении при его прохождении по каналу), «фильтры» (преобразователи сообщения, когда оно достигает шифратора или покидает Дешифратор) и др. Основным недостатком этой модели явилась недооценка других подходов в изучении проблемы общения;
	\item модель коммуникационного обмена, которая включала:
		\subitem a) коммуникационные условия;
		\subitem б) коммуникационное поведение;
		\subitem в) коммуникационные ограничения выбора стратегии общения;
		\subitem г) критерии интерпретации, определяющие и направляющие способы восприятия и оценки людьми своего поведения по отношению друг к другу.
\end{enumerate}

Интеракциональные подходы—рассматривают общение как ситуацию совместного присутствия, которая взаимно устанавливается и поддерживается людьми при помощи различных форм поведения и внешних атрибутов (внешности, предметов, обстановки и т.п.).В рамках интеракционных подходов было разработано пять моделей организации общения:
\begin{enumerate}
	\item лингвистическая модель, согласно которой все взаимодействия образуются и комбинируются из 50-60 элементарных движений и поз тела человека, а поведенческие акты, сформированные из этих единиц, организуются по принципу организации звуков в словах;
	\item модель социального навыка основывается на идее научения общению в самом общении;
	\item равновесная модель предполагает, что любое изменение поведения обычно компенсируется другим изменением, и наоборот (например, диалог — монолог, сочетание вопросов и ответов);
	\item программная модель социального взаимодействия предполагает, что общая структура межличностного взаимодействия порождается благодаря действию по крайней мере трех видов программ:
		\subitem a) программы, имеющей дело с простой координацией движений;
		\subitem б) программы, контролирующей изменение видов активности индивидов в ситуации, когда возникают помехи или неопределенности;
		\subitem в) программы, управляющей комплексной задачей мета-общения.
		
	Эти программы усваиваются индивидами по мере научения и позволяют организовывать разнородный поведенческий материал. Они «запускаются» в зависимости от содержательного контекста конкретной ситуации, задачи и социальной организации;
	\item системная модель рассматривает взаимодействие как конфигурацию систем поведения, управляющих обменом речевых высказываний и использованием пространства и территории взаимодействия.
\end{enumerate}

Реляционный подход строится на том, что общение — это система взаимоотношений, которые люди развивают друг с другом, с обществом и средой обитания, в которой они живут. Под информацией же понимается любое изменение какой-либо части этой системы, вызывающее изменение других частей. Люди, животные или другие организмы являются неотъемлемой частью процесса общения с момента рождения до момента смерти.

\section{Структура общения}
В структуре общения различают: 
\begin{enumerate}
	\item коммуникативную сторону;
	\item интерактивную сторону;
	\item перцептивную сторону \cite{28}.
\end{enumerate}

Коммуникативная сторона общения выражается в обмене информацией между людьми. 

Особенности процесса обмена информацией в процессе человеческого общения:
\begin{enumerate}
	\item происходит не только передача информации, но и ее формирование, уточнение и развитие;
	\item обмен информацией сочетается с отношением людей друг к другу;
	\item происходит взаимное влияние и воздействие людей друг на друга;
	\item  коммуникативное влияние людей друг на друга возможно только при совпадении систем кодификации у коммуникатора (отправителя) и реципиента (принимающего);
	\item возможно возникновение специфических коммуникативных барьеров социального и психологического характера.
\end{enumerate}

Структурные компоненты общения как коммуникативной деятельности:
\begin{enumerate}
	\item субъект общения — коммуникатор;
	\item объект общения — реципиент;
	\item предмет общения — содержательная часть отправляемой информации;
	\item действия общения—единицы коммуникативной активности;
	\item средства общения — операции, с помощью которых осуществляются действия общения;
	\item продукт общения — образование материального и духовного характера как итог общения.
\end{enumerate}

Интерактивная сторона общения проявляется во взаимодействии людей друг с другом, т.е. обмене информацией, побуждениями, действиями. Цель взаимодействия состоит в удовлетворении своих потребностей, интересов, реализации целей, планов, намерений.

Типы взаимодействия:
\begin{enumerate}
	\item положительные—взаимодействия, направленные на организацию совместной деятельности: кооперация; согласие; приспособление; ассоциация; 
	\item отрицательные — взаимодействия, направленные на нарушение совместной деятельности, создание для нее препятствий: конкуренция; конфликт; оппозиция; диссоциация.
\end{enumerate}

Факторы, влияющие на тип взаимодействия:
\begin{enumerate}
	\item степень единства подходов к решению проблем;
	\item понимание обязанностей и прав;
	\item способы решения возникающих проблем и др.
\end{enumerate}

Перцептивная сторона общения выражается в процессе восприятия, изучения и оценки партнерами друг друга.

Структурные элементы социальной перцепции:
\begin{enumerate}
	\item субъект межличностного восприятия — тот, кто воспринимает (изучает) в процессе общения;
	\item объект восприятия — тот, кого воспринимают (познают) в процессе общения;
	\item процесс познания — включает познание, обратную связь, элементы коммуникации.
\end{enumerate}

В процессе общения человек выступает сразу в двух ипостасях: как объект и как субъект познания.

Факторы, влияющие на процесс межличностного восприятия:
\begin{enumerate}
	\item особенности субъекта: половые различия (женщины точнее идентифицируют эмоциональные состояния, достоинства и недостатки личности, мужчины—уровень интеллекта); возраст, темперамент (экстраверты точнее воспринимают, интроверты — оценивают); социальный интеллект (чем выше уровень социальных и общих знаний, тем точнее оценка при восприятии); психическое состояние; состояние здоровья; установки—предшествующая оценка объектов восприятия; ценностные ориентации; уровень социально-психологической компетентности и т.д.
	\item особенности объекта: физический облик (антропологические — рост, телосложение, цвет кожи и т.д., физиологические — дыхание, кровообращение, функциональные — осанка, поза и походка и паралингвистические — мимика, жесты и телодвижения); социальный облик: социальная роль, внешний облик, проксемические особенности общения (расстояние и расположение общающихся), речевые и экстралингвистические характеристики (семантика, грамматика и фонетика), деятельностные особенности;
	\item отношения между субъектом и объектом восприятия;
	\item ситуация, в которой происходит перцепция.
\end{enumerate}

\section{Виды общения}
Виды общения по средствам: 
\begin{enumerate}
	\item вербальное общение — осуществляется посредством речи и является прерогативой человека. Оно предоставляет человеку широкие коммуникативные возможности и гораздо богаче всех видов и форм невербального общения, хотя в жизни не может полностью его заменить;
	\item невербальное общение происходит с помощьюмимики, жестов и пантомимики, через прямые сенсорные или телесные контакты (тактильные, зрительные, слуховые, обонятельные и другие ощущения и образы, получаемые от другого лица). Невербальные формы и средства общения присущи не только человеку, но и некоторым животным (собакам, обезьянам и дельфинам). В большинстве случаев невербальные формы и средства общения человека являются врожденными. Они позволяют людям взаимодействовать друг с другом, добиваясь взаимопонимания на эмоциональном и поведенческом уровнях. Важнейшей невербальной составляющей процесса общения является умение слушать \cite{13}.
\end{enumerate}
 
Виды общения по целям:
\begin{enumerate}
	\item биологическое общение связано с удовлетворением основных органических потребностей и необходимо для поддержания, сохранения и развития организма;
	\item социальное общение направлено на расширение и укрепление межличностных контактов, установление и развитие интерперсональных отношений, личностного роста индивида.
\end{enumerate}

Виды общения по содержанию:
\begin{enumerate}
	\item материальное — обмен предметами и продуктами деятельности, которые служат средством удовлетворения их актуальных потребностей;
	\item когнитивное — передача информации, расширяющей кругозор, совершенствующей и развивающей способности;
	\item кондиционное — обмен психическими или физиологическими состояниями, оказание влияния друг на друга, рассчитанное на то, чтобы привести человека в определенное физическое или психическое состояние;
	\item деятельностное — обмен действиями, операциями, умениями, навыками;
	\item мотивационное общение состоит в передаче друг другу определенных побуждений, установок или готовности к действиям в определенном направлении. 
\end{enumerate}

По опосредованности:
\begin{enumerate}
	\item непосредственное общение — происходит с помощью естественных органов, данных живому существу природой: руки, голова, туловище, голосовые связки и т.п.;
	\item опосредствованное общение — связано с использованием специальных средств и орудий для организации общения и обмена информацией (природных(палка, брошенный камень, след на земле и т. д.) или культурных предметов (знаковые системы, записи символов на различных носителях, печать, радио, телевидение и т. п.));
	\item прямое общение строится на основе личных контактов и непосредственного восприятия друг другом общающихся людей в самом акте общения (например, телесные контакты, беседы людей и т.д.);
	\item косвенное общение происходит через посредников, которыми могут быть другие люди (например, переговоры между конфликтующими сторонами на межгосударственном, межнациональном, групповом, семейном уровнях).
\end{enumerate}

Другие виды общения:
\begin{enumerate}
	\item деловое общение — общение, целью которого является достижение какого-либо четкого соглашения или договоренности;
	\item воспитательное общение — предполагает целенаправленное воздействие одного участника на другого с достаточно четким представлением желаемого результата;
	\item диагностическое общение — общение, целью которого является формулировка определенного представления о собеседнике или получение от него какой-либо информации (таково общение врача с пациентом и т.п.);
	\item интимно-личностное общение — возможно при заинтересованности партнеров в установлении и поддержании доверительного и глубокого контакта, возникает между близкими людьми и в значительной степени является результатом предшествующих взаимоотношений.
\end{enumerate}

\section{Формы общения}
\begin{enumerate}
	\item монологическая — когда только одному из партнеров отводится роль активного участника, а другому — пассивного исполнителя (например, лекция, нотация и т.д.);
	\item диалоговая — характерно сотрудничество участников — собеседников или партнеров по общению (например, беседа, разговор);
	\item полилогическая — многостороннее общение, которое носит характер борьбы за коммуникативную инициативу.
\end{enumerate}

\section{Уровни общения}
В зарубежной и отечественной психологии имеются различные взгляды на уровни общения \cite{28}. 

Уровни общения по Б.Г. Ананьеву:
\begin{enumerate}
	\item микроуровень—состоит из самых мелких элементов межличностногообщения с ближайшим окружением, с которым человек живет и чаще всего вступает в контакт (семья, друзья);
	\item мезоуровень — общение на уровне школы, производственной бригады и т.д.;
	\item макроуровень — включает такие крупные структуры, как управление и торговля. 
\end{enumerate}

Уровни общения по Э. Берну:
\begin{enumerate}
	\item ритуалы — это определенный порядок действий, которым совершается и закрепляется обычай;
	\item времяпрепровождение (просмотр телевизора, чтение книг, танцы и т.д.);
	\item игры—виды деятельности, результатом которых не становится производство какого-либо продукта;
	\item близость — интимные отношения;
	\item деятельность—специфический вид активности человека, направленный на познание и преобразование окружающего мира. 
\end{enumerate}

Наиболее распространенной в отечественной психологии является следующая уровневая система:
\begin{enumerate}
	\item примитивный уровень — предполагает реализацию схемы общения, в которой собеседник не партнер, а нужный или мешающий предмет. При этом фазы контакта исполняются в пристройке сверху или (с откровенно сильным партнером) снизу. Подобный уровень общения предлагается в состоянии опьянения, озлобления, в состоянии конфликта и т.д.;
	\item манипулятивный уровень—реализуется схема «партнер — соперник» в игре, которую непременно надо выиграть, причем выигрыш — выгода (материальная, житейская или психологическая). При этом манипулятор улавливает и пытается использовать слабые места партнера;
	\item стандартизованный уровень — общение, основанное на стандартах, когда один из партнеров (или оба) не желают контакта, но без него не обойтись;
	\item конвенциональный уровень — уровень обычного равноправного человеческого общения в рамках принятых правил поведения. Этот уровень требует от партнеров высокой культуры общения, которое может рассматриваться как искусство и для овладения которым иному человеку приходится годами работать над собой. Он является оптимальным для разрешения личных и межличностных проблем в человеческих контактах;
	\item игровой уровень — характеризуется так же, как конвенциональный, но с повышенной положительной направленностью на партнера, интересом к нему и желанием породить Подобный же интерес к себе со стороны партнера. В игре главное — заинтриговать, заинтересовать партнера. На этом уровне больше ценится возникшая человеческая связь, а не информативная компонента общения. Идеален для преподавательской деятельности;
	\item уровень делового общения — по сравнению с конвенциональным уровнем предполагает повышенную направленность на партнера как на участника коллективной деятельности. Главным на этом уровне является степень умственной и деловой активности партнера, его включенность в общую задачу. Идеален для групповой деятельности, для мозговых штурмов и т.д.;
	\item духовный уровень — высший уровень человеческого общения, для которого характерно взаиморастворение в партнере, высокая спонтанность мысли и чувства, предельная свобода самовыражения; партнер воспринимается как носитель духовного начала, и это начало пробуждает в нас чувство, которое сродни благоговению. 
\end{enumerate}

\section{Функции и средства общения}
Функции общения — это роли и задачи, которые выполняет общение в процессе социального бытия человека:  
\begin{enumerate}
	\item информационно-коммуникативная функция состоит в обмене информацией между индивидами. Составными элементами общения являются: коммуникатор (передает информацию), содержание сообщения, реципиент (принимает сообщение). Эффективность передачи информации проявляется в понимании информации, ее принятии или непринятии, усвоении. Для осуществления информационно-коммуникативной функции необходимо наличие единой или сходной системы кодификации/декодификации сообщений. Передача любой информации возможна посредством различных знаковых систем;
	\item побудительная функция—стимуляция активности партнеров для организации совместных действий;
	\item интегративная функция — функция объединения людей;
	\item функция социализации — общение способствуетвыработке навыков взаимодействия человека в обществе по принятым в нем нормам и правилам;
	\item координационная функция — согласование действий при осуществлении совместной деятельности;
	\item функция понимания—адекватное восприятие и понимание информации;
	\item регуляционно-коммуникативиая (интерактивная) функция общения направлена на регуляцию и коррекцию поведения при непосредственной организации совместной деятельности людей в процессе их взаимодействия;
	\item аффективно-коммуникативная функция общения состоит в воздействии на эмоциональную сферу человека, которое может быть целенаправленным или непроизвольным \cite{28}. 
\end{enumerate}

Средства общения — способы кодирования, передачи, переработки и расшифровки информации, передаваемой в процессе общения. Они бывают вербальные и невербальные. Вербальные средства общения—слова с закрепленными за ними значениями. Слова могут быть произнесены вслух (устная речь), написаны (письменная речь), заменены жестами у слепых или произнесены про себя. Устная речь является более простой и экономичной формой вербальных средств.

Она разделяется на:
\begin{enumerate}
	\item диалогическую речь, в которой принимают участие два собеседника;
	\item монологическую речь — речь, которую произносит один человек.
\end{enumerate}

Письменная речь применяется при невозможности устного общения или когда необходима точность, выверенность каждого слова.

Невербальные средства общения — знаковая система, которая дополняет и усиливает вербальную коммуникацию, а иногда и заменяет ее. С помощью невербальных средств общения передается около 55-65\% информации.

К невербальным средствам общения относятся:
\begin{enumerate}
	\item визуальные средства:
		\subitem a) кинестетические средства—это зрительно воспринимаемые движения другого человека, выполняющие выразительно-регулятивную функцию в общении. К кинесике относятся выразительные движения, проявляющиеся в мимике, позе, жесте, взгляде, походке;
		\subitem б) направление взгляда и визуальный контакт;
		\subitem в) выражение лица;
		\subitem г) выражение глаз;
		\subitem д) поза — расположение тела в пространстве («нога на ногу», перекрещенные руки, ноги и т.д.);
		\subitem е) дистанция (расстояние до собеседника, угол поворота к нему, персональное пространство);
		\subitem ж) кожные реакции (покраснение, испарина);
		\subitem з) вспомогательные средства общения (особенности телосложения (половые, возрастные)) и средства их преобразования (одежда, косметика, очки, украшения, татуировка, усы, борода, сигарета и т. п.);
	\item акустические (звуковые):
		\subitem a) связанные с речью (громкость, тембр, интонация, тон, высота звука, ритм, речевые паузы и их локализация в тексте);
		\subitem б) не связанные с речью (смех, скрежет зубов, плач, кашель, вздохи и т.п.);
	\item тактильные — связанные с прикосновением:
		\subitem a) физическое воздействие (ведение слепого за руку и др.);
		\subitem б) такевика (пожатие руки, хлопание по плечу). 
\end{enumerate}
