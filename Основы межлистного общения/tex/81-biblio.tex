\addcontentsline{toc}{chapter}{Список использованных источников}
\UnnamedStructuredChapter{Список использованных источников}
\bibliographystyle{utf8gost705u}
\begin{thebibliography}{1}
%	\bibitem{1}
%	Андреева Г.М. Социальная психология. Учебник для высших учебных заведений / Г.М. Андреева. - М.: Аспект Пресс, 2008. - 378 с. 
	
%	\bibitem{2}
%	Андриенко Е.В. Социальная психология: учебное пособие для студентов пед.вузов. М.: 2007.
	
	\bibitem{3}
	Аскевис-Леерпе, Ф. Психология: краткий курс/Ф. Аскевис-Леерпе, К. Барух, А. Картрон; пер. с франц. М.Л. Карачун. - М.: АСТ: Астрель, 2006. - 155 с.
	
	\bibitem{4}
	Бодалев А.А. Психология межличностного общения. Рязань, 1994.
	
	\bibitem{28}
	Челдышова Н. Б. Шпаргалка по социальной психологии. М.: Экзамен, 2009.
	
	\bibitem{5}
	Бодалев А.А. Психология общения. Избранные психологические труды. - 3-е изд., перераб. и допол. - М.: Издательство Московского психолого социального института; Воронеж: Издательство НПО "МОДЭК", 2002.- 320с.
	
%	\bibitem{6}
%	Большая энциклопедия психологических тестов. М.: Изд-во Эксмо, 2005. - 416 с.
	
%	\bibitem{7}
%	Вердербер, Р., Вердербер, К. Психология общения. - СПб.: прайм - ЕВРОЗНАК, 2003. - 320 с.
	
%	\bibitem{8}
%	Выготский Л.С. Психология развития человека. М.: ЭКСМО, 2003.
	
	\bibitem{9}
	Глейтман Г. Фридлунд А., Райсберг Д. Основы психологии. Спб.: Речь, 2001.
	
%	\bibitem{10}
%	Горянина В.А. Психология общения: Учеб.пособие для студ. Высш. Учеб. Заведений. - М.:Издательский центр "Академия", 2002. - 416 с.
	
%	\bibitem{11}
%	Дружинин В.Н. Структура и логика психологического исследования. М.: ИП РАН, 1994.
	
%	\bibitem{12}
%	Ермолаев О.Ю. Математическая статистика для психологов: Учебник /О.Ю. Ермолаев. – 2-е изд., испр. – М.: Московский психолого-социальный институт: Флинта, 2003. – 336 с. 
	
	\bibitem{13}
	Емельянов Ю.Н., Кузьмин Е.С. Теоретические и методические основы социально-психологического тренинга. Л.: ЛГУ, 1983. - 103 с.
	
%	\bibitem{14}
%	Краткий психологический словарь /Сост. Л.А. Карпенко; Под. Общ. ред. А.В . Петровского, М.Г. Ярошевского. - М.: Политиздат, 1985. - 431 с. 
	
%	\bibitem{15}
%	Крысько В.Г. Социальная психология: словарь-справочник. - Мн.: Харвест, 2004. - 688 с.	
	
%	\bibitem{16}
%	Крысько В.Г. Социальная психология: Учеб. для вузов. 2-е изд. - СПб.: Питер, 2006. - 432 с. 
	
	\bibitem{17}
	Леонтьев А.Н. Деятельность, сознание, личность. М.: Смысл: Издательский центр «Академия», 2006. 
	
%	\bibitem{18}
%	Мокшанцев Р.И., Мокшанцева А.В. Социальная психология: учеб. Пособие для вузов. М.: 2001.
	
%	\bibitem{19}
%	Прутченков А.С. Социально-психологический тренинг межличностного общения. М., 1991 - 45 с. 
	
%	\bibitem{20}
%	Психологические тесты /Под ред. А.А. Карелина: В 2 т. - М.: Гуманит. изд. центр ВЛАДОС, 2003. - Т.2. - 248 с. 
	
%	\bibitem{21}
%	Психология и педагогика военного управления. Учебно-методическое пособие. /Изд. ВВИА им. В.В. Жуковского, 1992. 
	
%	\bibitem{22}
%	Ребер А. Большой толковый психологический словарь. В 2-х томах. М.: Вече, АСТ, 2000. 
	
%	\bibitem{23}
%	Семечкин, Н.И. Социальная психология: учебник для вузов. - СПб.: Питер, 2004. - 376 с.
	
%	\bibitem{24}
%	Социальная психология: учебное пособие для вузов /Под ред. А.А. Журавлева. М.: 2003.
	
%	\bibitem{25}
%	Справочник практического психолога. Психодиагностика/под.общ.ред. С.Т.Посоховой. - М.: АСТ; СПб.: Сова, 2005. - 671, [1] с.:ил.
	
%	\bibitem{26}
%	Фолкэн Чак Т. Психология - это просто / Пер. с англ. Р.Муртазина. - М.: ФАИР-ПРЕСС, 2001. - 640 с. 
	
%	\bibitem{27}
%	Шевандрин Н.И. Социальная психология в образовании. М. 1995.
	
\end{thebibliography}
