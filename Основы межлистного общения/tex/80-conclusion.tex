\StructuredChapter{Заключение}
Рассматривая межличностные отношения можно сделать вывод, что межличностные отношения – субъективно переживаемые связи между людьми, объективно проявляющиеся в характере и способах межличностного взаимодействия, т.е. взаимных влияний, оказываемых людьми друг на друга в процессе их совместной деятельности и общения. 

Межличностные отношения – это система установок, ориентаций и ожиданий членов группы относительно друг друга, определяющихся содержанием и организацией совместной деятельности и ценностями, на которых основывается общение людей. при этом возможно рассогласование субъективно переживаемых и объективно существующих связей индивида с другими людьми. В группах разного уровня развития. 

Межличностные отношения различаются не только количественно, но и качественно. Так, в коллективе они составляют сложную иерархическую структуру, которая развивается по мере включения его в общественно значимую деятельность. Экспериментальное исследование межличностных отношений осуществляется социальной психологией с помощью специальных методик: Социометрия, Референтометрический метод, Методы исследования личности. Чаще всего в практике используется Социометрической метод Дж. Морено. 

Общение можно охарактеризовать как сложный, многоплановый процесс установления и развития контактов между людьми, порождаемый потребностями совместной деятельности и включающий в себя обмен информацией, выработку единой стратегии взаимодействия, восприятие и понимание другого человека. Соответственно в общении различаются три стороны: коммуникативная, интерактивная и перцептивная. Где коммуникативная сторона общения связана с выявлением информационного процесса между людьми как активными субъектами, т.е. с учетом отношений между партнерами, их установок, целей, намерений, что приводит не просто к «движению» информации, но к уточнению и обогащению тех знаний, сведений, мнений, которыми обмениваются люди. Средствами коммуникативного процесса являются различные знаковые системы, прежде всего речь, а также оптико-кинетическая система знаков (жесты, мимика, пантомимика), пара- и экстралингвистическая системы (интонация, неречевые вкрапления в речь, например, паузы), система организации пространства и времени коммуникации, система «контакта глазами». Интерактивная сторона общения представляет собой построение общей стратегии взаимодействия. Различают ряд типов взаимодействия между людьми, прежде всего кооперацию и конкуренцию. Перцептивная сторона общения включает в себя процесс формирования образа другого человека, что достигается «прочтением» за физическими характеристиками человека, его психологических свойств и особенностей его поведения. Основными механизмами познания другого человека являются идентификация и рефлексия. 

Самой важной психической составляющей отношения оказывается мотивационно-эмоциональный компонент, который сигнализирует о валентности отношения – положительной, отрицательной, противоречивой или безразличной. 

Особой проблемой при изучении взаимозависимостей общения и отношения является установление соответствия характера и способов выражения отношения; так же влияет социальное значение и система ценностей. 
